\documentclass[12pt]{article}

\usepackage[top = 1in, bottom = 1in, right = 1in, left = 1in] {geometry}
\usepackage{mathptmx, tikz, nopageno}
\usetikzlibrary{calc}

\title{\color[HTML]{09447a}\textbf{Importance of Cauchy Distribution \\ in the Field of Statistics}}
\author{\textbf{Anik Chakraborty, 443}\\ \textbf{Registration No.: A01-1112-0838-19} \\ \textbf{Session: 2019 - '22}}
\date{\textbf{Supervisor: Prof. Ayan Chandra}}

\begin{document}
	\maketitle
	\begin{tikzpicture}
		[remember picture, overlay] \draw[line width = 2pt] 
		($(current page.north west) + (0.5in,-0.5in)$) rectangle ($(current page.south east) + (-0.5in,0.5in)$);
	\end{tikzpicture}

	\color[HTML]{1a5c8f}\section*{Synopsis}
	\color[HTML]{1a76bd}\subsection*{Objectives}
	\paragraph{} \color{black} Cauchy distribution has an important impact on the Statistical society, specially in the Estimation Theory. The distribution with density $ \frac{1}{\pi(1+x^2)} $ has some peculiar properties which was first observed by Simeon Denis Poisson. This leads to counter examples to some well known and widely accepted results and theories in Statistics. 
	\paragraph{} The Cauchy distribution has also some unique features for which this always seeks a special attention of the Statisticians and make them indulge in rigorous studies of its characteristics. `Apart from the Normal distribution why Cauchy distribution is needed' - to study this, is another aspect of this project. 
	
	\color[HTML]{1a76bd}\subsection*{Methodology}
	\paragraph{} \color{black} To study the above mentioned topics, we focus on two broad sections - 
	\begin{itemize}
		\item Difference between Normal and Cauchy population \\ [5pt] $-$ Here, we will run a simulation program to observe the behaviour of the tests under the normality assumptions and sample from a Cauchy population
		\item Cauchy Distribution in Estimation Theory \\ [5pt] $-$ Here, we will check the consistency, Maximum Likelihood Estimation and the concepts of Cramer-Rao Inequality under the Cauchy setup.
	\end{itemize}
\vspace{28pt}
\textbf{Student's Signature:} \line(1,0){150} \hfill \textbf{Date:} \line(1,0){80}
\end{document}